\documentclass[10pt]{article}

% The following command leaves more space between lines.  That's great
% when correcting drafts.  When you comment it out, however, the
% output looks much nicer.
%
\linespread{1.0}

\usepackage{amsmath}
\usepackage{amssymb}
\usepackage{graphicx}
\usepackage{epsfig}
\usepackage{latexsym}
\usepackage{amsthm}

\usepackage{mathrsfs}

\usepackage{multicol}



\usepackage[colorlinks,citecolor=blue]{hyperref}

\usepackage[latin1]{inputenc}

\usepackage{tikz-cd}
\usepackage{pgfplots}

%\usepackage{3dplot}

\usetikzlibrary{matrix,arrows,decorations.pathmorphing}


\usepackage[scale=0.8]{geometry}


%\usepackage{umoline}\setlength{\UnderlineDepth}{1pt}
%\usepackage[linktocpage=true]{hyperref}

\input xy
\xyoption{all}


%\addtolength{\hoffset}{-.5in}
%\addtolength{\textwidth}{1in}
%\setlength{\parindent}{.5in}
%\setlength{\textheight}{9.5in} \setlength{\topmargin}{-2cm}


\pagestyle{myheadings}\parindent 0em




\usepackage[latin1]{inputenc}


%------------------copy and posted code from the internets-------------

%\numberwithin{equation}{section} % comment out when neccessary

\newtheorem{theorem}[equation]{Theorem}
\newtheorem{lemma}[equation]{Lemma}
\newtheorem{proposition}[equation]{Proposition}
\newtheorem{corollary}[equation]{Corollary}


\theoremstyle{definition}
\newtheorem{definition}[equation]{Definition}
\newtheorem{example}[equation]{Example}
\newtheorem{remark}[equation]{Remark}
\newtheorem{problem}[equation]{Problem}



\newcommand{\R}[1]{\mathbb{R}^{#1}}
\newcommand{\C}[1]{\mathbb{C}^{#1}}
\newcommand{\Z}[1]{\mathbb{Z}^{#1}}
\newcommand{\K}[1]{\mathbb{K}^{#1}}
\newcommand{\embed}[0]{\hookrightarrow}
\newcommand{\TT}[4]{\begin{tabular}{| c | c |}\hline $#1$ & $#2$ \\ \hline $#3$ & $#4$ \\ \hline\end{tabular}} %goddamn it
\newcommand{\partd}[2]{\frac{\partial #1}{\partial #2}}
\newcommand{\limit}[2]{\displaystyle{ \lim_{#1 \to #2}}}
\newcommand{\vectornorm}[1]{\left|\left|#1\right|\right|}
\newcommand{\Ker}[0]{\text{\textnormal{Ker}}}
\newcommand{\Hom}[0]{\text{\textnormal{Hom}}}
\newcommand{\circled}[1]{\tikz[baseline=(char.base)]{
            \node[shape=circle,draw,inner sep=2pt] (char) {#1};}}


\newcommand{\T}{\rotatebox[origin=c]{180}{$\scriptscriptstyle \perp $}}
\newcommand{\x}{\textbf{x}}
\newcommand{\y}{\textbf{y}}
\newcommand{\supp}{\text{\textnormal{supp}}}
\newcommand{\csupp}{\text{\textnormal{cosupp}}}
\newcommand{\found}{\text{\textnormal{found}}}
\newcommand{\roof}{\text{\textnormal{roof}}}

\newcommand{\bcup}{\displaystyle\bigcup}
\newcommand{\bcap}{\displaystyle\bigcap}
\newcommand{\dsum}{\displaystyle\sum}
\newcommand{\dint}{\displaystyle\int}





\begin{document}
%

{\bf Name:} \hrulefill\hrulefill\hrulefill\\
{\bf M143} \qquad \qquad \\
{\bf Chapter 11 Help}\\ %(look familiar??)\\
%Show all work for full/partial credit.
%---------------- End of the document ---------------

What is the derivative of $g(x)=\sqrt{11x}$?  There are two ways to do this.

\begin{enumerate}
\item Using the limit definition:

\begin{eqnarray*}
g'(x)&=&\limit{h}{0}\frac{\sqrt{11(x+h)}-\sqrt{11x}}{h}\\
&=&\limit{h}{0}\frac{\sqrt{11x+11h}-\sqrt{11x}}{h}\\
&=&\limit{h}{0}\frac{\sqrt{11x+11h}-\sqrt{11x}}{h}\frac{\sqrt{11x+11h}+\sqrt{11x}}{\sqrt{11x+11h}-\sqrt{11x}}\\
&=&\limit{h}{0}\frac{11x+11h-11x}{h(\sqrt{11x+11h}+\sqrt{11x})}\\
&=&\limit{h}{0}\frac{11h}{h(\sqrt{11x+11h}+\sqrt{11x})}\\
&=&\limit{h}{0}\frac{11}{(\sqrt{11x+11h}+\sqrt{11x})}\\
&=&\frac{11}{(\sqrt{11x+0}+\sqrt{11x})}\\
&=&\frac{11}{2\sqrt{11x}}
\end{eqnarray*}

\item On the other hand, if you've started reading into Chapter 12 or you know something about derivatives:

\begin{eqnarray*}
g(x)&=&\sqrt{11x}=(11x)^{\frac{1}{2}}\\
g'(x)&=&\frac{1}{2}\cdot(11x)^{-\frac{1}{2}}\cdot\frac{d}{dx}[11x],\ \text{by the chain rule:}\\
&=&\frac{1}{2}\cdot(11x)^{-\frac{1}{2}}\cdot11,\ \text{then we rewrite this a bit}:\\
&=&\frac{1}{2}\frac{1}{\sqrt{11x}}11\\
&=&\frac{11}{2\sqrt{11x}}.
\end{eqnarray*}

\end{enumerate}





















\end{document}
