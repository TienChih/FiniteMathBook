\documentclass[10pt]{article}

% The following command leaves more space between lines.  That's great
% when correcting drafts.  When you comment it out, however, the
% output looks much nicer.
%
\linespread{1.0}

\usepackage{amsmath}
\usepackage{amssymb}
\usepackage{graphicx}
\usepackage{epsfig}
\usepackage{latexsym}
\usepackage{amsthm}

\usepackage{mathrsfs}

\usepackage{multicol}



\usepackage[colorlinks,citecolor=blue]{hyperref}

\usepackage[latin1]{inputenc}

\usepackage{tikz-cd}
\usepackage{pgfplots}

%\usepackage{3dplot}

\usetikzlibrary{matrix,arrows,decorations.pathmorphing}


\usepackage[scale=0.8]{geometry}


%\usepackage{umoline}\setlength{\UnderlineDepth}{1pt}
%\usepackage[linktocpage=true]{hyperref}

\input xy
\xyoption{all}


%\addtolength{\hoffset}{-.5in}
%\addtolength{\textwidth}{1in}
%\setlength{\parindent}{.5in}
%\setlength{\textheight}{9.5in} \setlength{\topmargin}{-2cm}


\pagestyle{myheadings}\parindent 0em




\usepackage[latin1]{inputenc}


%------------------copy and posted code from the internets-------------

%\numberwithin{equation}{section} % comment out when neccessary

\newtheorem{theorem}[equation]{Theorem}
\newtheorem{lemma}[equation]{Lemma}
\newtheorem{proposition}[equation]{Proposition}
\newtheorem{corollary}[equation]{Corollary}


\theoremstyle{definition}
\newtheorem{definition}[equation]{Definition}
\newtheorem{example}[equation]{Example}
\newtheorem{remark}[equation]{Remark}
\newtheorem{problem}[equation]{Problem}



\newcommand{\R}[1]{\mathbb{R}^{#1}}
\newcommand{\C}[1]{\mathbb{C}^{#1}}
\newcommand{\Z}[1]{\mathbb{Z}^{#1}}
\newcommand{\K}[1]{\mathbb{K}^{#1}}
\newcommand{\embed}[0]{\hookrightarrow}
\newcommand{\TT}[4]{\begin{tabular}{| c | c |}\hline $#1$ & $#2$ \\ \hline $#3$ & $#4$ \\ \hline\end{tabular}} %goddamn it
\newcommand{\partd}[2]{\frac{\partial #1}{\partial #2}}
\newcommand{\limit}[2]{\displaystyle{ \lim_{#1 \to #2}}}
\newcommand{\vectornorm}[1]{\left|\left|#1\right|\right|}
\newcommand{\Ker}[0]{\text{\textnormal{Ker}}}
\newcommand{\Hom}[0]{\text{\textnormal{Hom}}}
\newcommand{\circled}[1]{\tikz[baseline=(char.base)]{
            \node[shape=circle,draw,inner sep=2pt] (char) {#1};}}


\newcommand{\T}{\rotatebox[origin=c]{180}{$\scriptscriptstyle \perp $}}
\newcommand{\x}{\textbf{x}}
\newcommand{\y}{\textbf{y}}
\newcommand{\supp}{\text{\textnormal{supp}}}
\newcommand{\csupp}{\text{\textnormal{cosupp}}}
\newcommand{\found}{\text{\textnormal{found}}}
\newcommand{\roof}{\text{\textnormal{roof}}}

\newcommand{\bcup}{\displaystyle\bigcup}
\newcommand{\bcap}{\displaystyle\bigcap}
\newcommand{\dsum}{\displaystyle\sum}
\newcommand{\dint}{\displaystyle\int}





\begin{document}
%

{\bf Name:} \hrulefill\hrulefill\hrulefill\\
{\bf M143} \qquad \qquad \\
{\bf Introduction to Systems of Linear Equations }\\ %(look familiar??)\\
%Show all work for full/partial credit.
%---------------- End of the document ---------------



\section{What Exactly is  a Solution to a System of Linear Equations?}

To answer the question that is the name of this section, we should kinda back up here and ask ``What is a solution to an equation?"\\

For Example, what is a solution to $y=2x+3$? After dealing with linear equations in Chapter 1, we should have some sense as to what $y=2x+3$ looks like, that is a line with slope 2 and $y$-intercept $(0,3)$.  But what IS this line?  It's the collection of all points $(x,y)$ that make the above equation true.  So $(0,3), (-1,1)$ and $(2,7)$ are all on this line because they make the equation true ($3=2(0)+3, 1=2(-1)+3, 7=2(2)+3$).  A point like $(1,1)$ is NOT on this line because $1\neq 2(1)+3$.  \\

So a solution to an equation is \textbf{ANY POINT THAT MAKES THE EQUATION TRUE}.\\

A \textit{System of Equations} is a collection of equations.  Thus, a solution to a system of equations is \textbf{ANY POINT THAT MAKES ALL THE EQUATIONS TRUE}.

\subsection{What are the Types of Solutions that we can have?}

So a solution to the system:

\begin{eqnarray*}
2x+3y&=&1\\
x+y&=&0
\end{eqnarray*}
is a collection of points that satisfy the first AND the second linear equalities.  The first equation describes the line: $y=-\frac{2}{3}x+\frac{1}{3}$, whereas the second describes the line $y=-x$.  Each of these equations represent lines, so each individually has \textbf{INFINITELY MANY SOLUTIONS}.  The question is, how many solutions do these lines have in common?\\

There are a number of ways to find out, but the most straight forward is simply to graph the system: \url{https://www.desmos.com/calculator/loxfnpb33x}, as you can see, although each line has infinitely many points, the only solution they have in common is the point $(-1,1)$.  To verify this, we note that $2(-1)+3(1)=1$ and $(-1)+1=0$, so both equations are satisfied.  This system has \textbf{A Unique Solution}\\

What else could happen when solving a system of equations?  Well, consider:


\begin{eqnarray*}
x+y&=&2\\
2x+2y&=&4.
\end{eqnarray*}

If we look at the graph of both lines, we only see one actual line: \url{https://www.desmos.com/calculator/ukkoiwu9gs}.  This is because both equations describe the same line.  After all, whenever $x+y=2$, it would have to be true that $2(x+y)=2\cdot2$ and thus $2x+2y=4$.  So ANY point that is a solution to one equation is a solution to the other, this system has \textbf{Infinitely Many Solutions}.\\
\newpage

Another possibility arises from the following example:


\begin{eqnarray*}
x+y&=&0\\
x+y&=&1.
\end{eqnarray*}

If we graph these lines side by side \url{https://www.desmos.com/calculator/cflf9d6fhs}, we can see that these lines are parallel and do not actually have any points in common.  After all, how many pairs of numbers $x$ and $y$ sum to both 0 AND 1?  It should be clear that this system has \textbf{NO SOLUTIONS}.\\

Of course, these have all been Systems of Linear Equations in 2 variables.  When we increase the number of variables, we increase the dimensionality of the potential solution space, and what it means to be ``linear" changes somewhat.  Nevertheless, these examples illustrate the 3 possible outcomes of a system of linear equations in any dimension:

\begin{itemize}
\item There is a unique Solution.
\item There are infinitely many Solutions.
\item There are no Solutions.
\end{itemize}



\subsection{How do we find the Solution(s)/know we have No Solutions?}

In 2 variables, it's simple enough we can just graph, as we did above.  But we should try to develop some algebraic techniques as well, since this will be harder to do when we increase the dimensions of our problems.\\


What we introduce here is the Echelon Method, the general goal is to take non-zero multiples of our equations and use them to eliminate variables in the other equations so that a solution to a variable can be made known.  This process should also reveal if there are infinitely many or no solutions.\\


\begin{itemize}
\item For the first example:

\begin{eqnarray*}
2x+3y&=&1\\
x+y&=&0
\end{eqnarray*}

We note that whatever the solutions to this are (pretend we didn't already solve this) they should be the same as the solutions to:

\begin{eqnarray*}
x+\frac{3}{2}y&=&\frac{1}{2}\\
x+y&=&0
\end{eqnarray*}

We've divided the first equation in half, but we haven't actually changed what points will make that equation true.  Then:

\begin{eqnarray*}
x+\frac{3}{2}y&=&\frac{1}{2}\\
x+y-(x+\frac{3}{2}y)&=&0-\frac{1}{2}
\end{eqnarray*}

Must also be true, since we can add and subtract the same values from 2 sides of an equation and preserve the solutions.  By doing this, we kill off the $x$ in the second equation and we can solve for $y$:

\begin{eqnarray*}
x+\frac{3}{2}y&=&\frac{1}{2}\\
-\frac{1}{2}y&=&-\frac{1}{2}
\end{eqnarray*}

So $y=1$ based on the second line.  

\begin{eqnarray*}
x+\frac{3}{2}y&=&\frac{1}{2}\\
y&=&1
\end{eqnarray*}


If we take $-\frac{3}{2}$ times the second line and add it to the first:

\begin{eqnarray*}
x+\frac{3}{2}y-\frac{3}{2}(y)&=&\frac{1}{2}-\frac{3}{2}(1)\\
y&=&1
\end{eqnarray*}

We obtain:

\begin{eqnarray*}
x&=&-1\\
y&=&1
\end{eqnarray*}

Which was the solution we found before.


\item So if we consider:

\begin{eqnarray*}
x+y&=&2\\
2x+2y&=&4.
\end{eqnarray*}

Just like before, we want to kill the $x$ in the second equation:

\begin{eqnarray*}
x+y&=&2\\
2x+2y-2(x+y)&=&4-2(2).
\end{eqnarray*}

Which gives us:

\begin{eqnarray*}
x+y&=&2\\
0&=&0.
\end{eqnarray*}

Well..., certainly that second line doesn't contribute much, zero is always equal to itself.  So the solutions are just whenever $x+y=2$, which happens for infinitely many possible pairs.

\item Lastly, when we look at:

\begin{eqnarray*}
x+y&=&0\\
x+y&=&1.
\end{eqnarray*}

We can try to kill the $x$ in the second line as before:

\begin{eqnarray*}
x+y&=&0\\
x+y-(x+y)&=&1-0.
\end{eqnarray*}

Where we get:

\begin{eqnarray*}
x+y&=&0\\
0&=&1.
\end{eqnarray*}

So all you have to do is find a pair $(x,y)$ so that $0=1$... which is not ever going to happen.  So this will never have a solution.


\end{itemize}

























\end{document}
