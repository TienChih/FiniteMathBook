\documentclass[10pt]{article}

% The following command leaves more space between lines.  That's great
% when correcting drafts.  When you comment it out, however, the
% output looks much nicer.
%
\linespread{1.0}

\usepackage{amsmath}
\usepackage{amssymb}
\usepackage{graphicx}
\usepackage{epsfig}
\usepackage{latexsym}
\usepackage{amsthm}

\usepackage{mathrsfs}

\usepackage{multicol}



\usepackage[colorlinks,citecolor=blue]{hyperref}

\usepackage[latin1]{inputenc}

\usepackage{tikz-cd}
\usepackage{pgfplots}

%\usepackage{3dplot}

\usetikzlibrary{matrix,arrows,decorations.pathmorphing}


\usepackage[scale=0.8]{geometry}


%\usepackage{umoline}\setlength{\UnderlineDepth}{1pt}
%\usepackage[linktocpage=true]{hyperref}

\input xy
\xyoption{all}


%\addtolength{\hoffset}{-.5in}
%\addtolength{\textwidth}{1in}
%\setlength{\parindent}{.5in}
%\setlength{\textheight}{9.5in} \setlength{\topmargin}{-2cm}


\pagestyle{myheadings}\parindent 0em




\usepackage[latin1]{inputenc}


%------------------copy and posted code from the internets-------------

%\numberwithin{equation}{section} % comment out when neccessary

\newtheorem{theorem}[equation]{Theorem}
\newtheorem{lemma}[equation]{Lemma}
\newtheorem{proposition}[equation]{Proposition}
\newtheorem{corollary}[equation]{Corollary}


\theoremstyle{definition}
\newtheorem{definition}[equation]{Definition}
\newtheorem{example}[equation]{Example}
\newtheorem{remark}[equation]{Remark}
\newtheorem{problem}[equation]{Problem}



\newcommand{\R}[1]{\mathbb{R}^{#1}}
\newcommand{\C}[1]{\mathbb{C}^{#1}}
\newcommand{\Z}[1]{\mathbb{Z}^{#1}}
\newcommand{\K}[1]{\mathbb{K}^{#1}}
\newcommand{\embed}[0]{\hookrightarrow}
\newcommand{\TT}[4]{\begin{tabular}{| c | c |}\hline $#1$ & $#2$ \\ \hline $#3$ & $#4$ \\ \hline\end{tabular}} %goddamn it
\newcommand{\partd}[2]{\frac{\partial #1}{\partial #2}}
\newcommand{\limit}[2]{\displaystyle{ \lim_{#1 \to #2}}}
\newcommand{\vectornorm}[1]{\left|\left|#1\right|\right|}
\newcommand{\Ker}[0]{\text{\textnormal{Ker}}}
\newcommand{\Hom}[0]{\text{\textnormal{Hom}}}
\newcommand{\circled}[1]{\tikz[baseline=(char.base)]{
            \node[shape=circle,draw,inner sep=2pt] (char) {#1};}}


\newcommand{\T}{\rotatebox[origin=c]{180}{$\scriptscriptstyle \perp $}}
\newcommand{\x}{\textbf{x}}
\newcommand{\y}{\textbf{y}}
\newcommand{\supp}{\text{\textnormal{supp}}}
\newcommand{\csupp}{\text{\textnormal{cosupp}}}
\newcommand{\found}{\text{\textnormal{found}}}
\newcommand{\roof}{\text{\textnormal{roof}}}

\newcommand{\bcup}{\displaystyle\bigcup}
\newcommand{\bcap}{\displaystyle\bigcap}
\newcommand{\dsum}{\displaystyle\sum}
\newcommand{\dint}{\displaystyle\int}





\begin{document}
%

{\bf Name:} \hrulefill\hrulefill\hrulefill\\
{\bf M143} \qquad \qquad \\
{\bf Derivatives of Exponential and Logarithmic functions}\\ %(look familiar??)\\
%Show all work for full/partial credit.
%---------------- End of the document ---------------

\section{Exponentials}

When we graph an exponential function, say an increasing one $f(x)=a^x, a>1$, we imagine a function that curves up towards infinity on the right and shallows out towards 0 on the left \url{https://www.desmos.com/calculator/51zxitu9xc}.  What then should it's derivatives look like?  Well it seems like the slopes are initially shallow, and then as we go forward the slopes start to increase, and they get progressively and progressively larger.  In other words {\bf the derivative is also an exponential function!} \url{https://www.desmos.com/calculator/etbmc1ttzo}.\\

Which one?  Let's first start with a special case, what is the derivative of $f(x)=e^x$?  $e$ is a special number, it has the peculiar property that:

$$\limit{h}{0}\frac{e^h-1}{h}=1.$$  The proof of this is a fairly sophisticated fact and it not trivial, but you can verify it by taking the slider $h$, dragging it towards 0 and seeing what happens: \url{https://www.desmos.com/calculator/4uv0zgguq9}.  Using this fact, we can deduce:

\begin{eqnarray*}
\frac{d}{dx}[e^x]&=&\limit{h}{0}\frac{e^{x+h}-e^x}{h}\\
&=&\limit{h}{0}\frac{e^xe^h-e^x}{h}\\
&=&e^x\limit{h}{0}\frac{e^h-1}{h}\\
&=&e^x\cdot1=e^x.
\end{eqnarray*}


So this explains why $e$ is such a significant number in mathematics, it is the unique number so that the exponential $e^x$ is it's own derivative \url{https://www.desmos.com/calculator/xelrg6t0ez}.   Every other $a^x$ differs from it's derivative slightly: \url{https://www.desmos.com/calculator/aqa6zw4vyz}, you can see that when $a$ is less than $e$ (about 2.714), it's derivative falls under the original function, and when $a>e$, it's derivative falls above the original.\\

So, to compute the actual derivative of $a^x$:

\begin{eqnarray*}
\frac{d}{dx}[a^x]&=&\frac{d}{dx}[(e^{\ln(a)})^x],\ \text{since exponentiating by $e$ and $\ln$ are inverses}\\
&=&\frac{d}{dx}[e^{\ln(a)x}]\\
&=&e^{\ln(a)x}\frac{d}{dx}[\ln(a)x],\ \text{by chain rule}\\
&=&e^{\ln(a)x}\ln(a)\cdot1\\
&=&a^x\ln(a).
\end{eqnarray*}


\section{Logarithms}

We can use the chain rule to find the derivative of log functions as well.  Let's first compute $\frac{d}{dx}[\ln(x)]$:

\begin{eqnarray*}
x&=&e^{\ln(x)}\\
\frac{d}{dx}[x]&=&\frac{d}{dx}[e^{\ln(x)}]\\
1&=&e^{\ln(x)}\frac{d}{dx}[\ln(x)]\\
1&=&x\frac{d}{dx}[\ln(x)]\\
\frac{d}{dx}[\ln(x)]&=&\frac{1}{x}.
\end{eqnarray*}

You can see it graphically verified here: \url{https://www.desmos.com/calculator/u89fjxup7q}.  For general $\log_a(x)$:


\begin{eqnarray*}
x&=&a^{\log_a(x)}\\
\frac{d}{dx}[x]&=&\frac{d}{dx}[a^{\log_a(x)}]\\
1&=&a^{\log_a(x)}\ln(a)\frac{d}{dx}[\log_a(x)]\\
1&=&x\ln(a)\frac{d}{dx}[\log_a(x)]\\
\frac{d}{dx}[\log_a(x)]&=&\frac{1}{x\ln(a)}.
\end{eqnarray*}


Verified graphically here as well: \url{https://www.desmos.com/calculator/znr3xikwcc}.  Drag $a$ around and see how everything matches up.

\section{Examples}


Let's try the derivative of $f(x)=e^{x}x^2$:

\begin{eqnarray*}
f'(x)&=&\frac{d}{dx}[e^xx^2]\\
&=_{PR}&\frac{d}{dx}[e^x]x^2+e^x\frac{d}{dx}[x^2]\\
&=&e^xx^2+e^x2x.
\end{eqnarray*}

Let $g(x)=\ln(x^2+1)$:

\begin{eqnarray*}
g'(x)&=&\frac{d}{dx}[\ln(x^2+1)]\\
&=_{CR}&\frac{1}{x^2+1}\frac{d}{dx}[x^2+1]\\
&=&\frac{2x}{x^2+1}.
\end{eqnarray*}

One last one, let $h(x)=2^{\frac{x}{x+1}}$\\

\begin{eqnarray*}
h'(x)&=&\frac{d}{dx}[2^{\frac{x}{x+1}}]\\
&=_{CR}&2^{\frac{x}{x+1}}\ln(2)\frac{d}{dx}[\frac{x}{x+1}]\\
&=_{QR}&2^{\frac{x}{x+1}}\ln(2)\frac{(x+1)\frac{d}{dx}[x]-x\frac{d}{dx}[x+1]}{(x+1)^2}\\
&=&\frac{2^{\frac{x}{x+1}}\ln(2)}{(x+1)^2}
\end{eqnarray*}














\end{document}
