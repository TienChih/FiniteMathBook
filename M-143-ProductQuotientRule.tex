\documentclass[10pt]{article}

% The following command leaves more space between lines.  That's great
% when correcting drafts.  When you comment it out, however, the
% output looks much nicer.
%
\linespread{1.0}

\usepackage{amsmath}
\usepackage{amssymb}
\usepackage{graphicx}
\usepackage{epsfig}
\usepackage{latexsym}
\usepackage{amsthm}

\usepackage{mathrsfs}

\usepackage{multicol}



\usepackage[colorlinks,citecolor=blue]{hyperref}

\usepackage[latin1]{inputenc}

\usepackage{tikz-cd}
\usepackage{pgfplots}

%\usepackage{3dplot}

\usetikzlibrary{matrix,arrows,decorations.pathmorphing}


\usepackage[scale=0.8]{geometry}


%\usepackage{umoline}\setlength{\UnderlineDepth}{1pt}
%\usepackage[linktocpage=true]{hyperref}

\input xy
\xyoption{all}


%\addtolength{\hoffset}{-.5in}
%\addtolength{\textwidth}{1in}
%\setlength{\parindent}{.5in}
%\setlength{\textheight}{9.5in} \setlength{\topmargin}{-2cm}


\pagestyle{myheadings}\parindent 0em




\usepackage[latin1]{inputenc}


%------------------copy and posted code from the internets-------------

%\numberwithin{equation}{section} % comment out when neccessary

\newtheorem{theorem}[equation]{Theorem}
\newtheorem{lemma}[equation]{Lemma}
\newtheorem{proposition}[equation]{Proposition}
\newtheorem{corollary}[equation]{Corollary}


\theoremstyle{definition}
\newtheorem{definition}[equation]{Definition}
\newtheorem{example}[equation]{Example}
\newtheorem{remark}[equation]{Remark}
\newtheorem{problem}[equation]{Problem}



\newcommand{\R}[1]{\mathbb{R}^{#1}}
\newcommand{\C}[1]{\mathbb{C}^{#1}}
\newcommand{\Z}[1]{\mathbb{Z}^{#1}}
\newcommand{\K}[1]{\mathbb{K}^{#1}}
\newcommand{\embed}[0]{\hookrightarrow}
\newcommand{\TT}[4]{\begin{tabular}{| c | c |}\hline $#1$ & $#2$ \\ \hline $#3$ & $#4$ \\ \hline\end{tabular}} %goddamn it
\newcommand{\partd}[2]{\frac{\partial #1}{\partial #2}}
\newcommand{\limit}[2]{\displaystyle{ \lim_{#1 \to #2}}}
\newcommand{\vectornorm}[1]{\left|\left|#1\right|\right|}
\newcommand{\Ker}[0]{\text{\textnormal{Ker}}}
\newcommand{\Hom}[0]{\text{\textnormal{Hom}}}
\newcommand{\circled}[1]{\tikz[baseline=(char.base)]{
            \node[shape=circle,draw,inner sep=2pt] (char) {#1};}}


\newcommand{\T}{\rotatebox[origin=c]{180}{$\scriptscriptstyle \perp $}}
\newcommand{\x}{\textbf{x}}
\newcommand{\y}{\textbf{y}}
\newcommand{\supp}{\text{\textnormal{supp}}}
\newcommand{\csupp}{\text{\textnormal{cosupp}}}
\newcommand{\found}{\text{\textnormal{found}}}
\newcommand{\roof}{\text{\textnormal{roof}}}

\newcommand{\bcup}{\displaystyle\bigcup}
\newcommand{\bcap}{\displaystyle\bigcap}
\newcommand{\dsum}{\displaystyle\sum}
\newcommand{\dint}{\displaystyle\int}





\begin{document}
%

{\bf Name:} \hrulefill\hrulefill\hrulefill\\
{\bf M143} \qquad \qquad \\
{\bf Product and Quotient rules of Derivatives}\\ %(look familiar??)\\
%Show all work for full/partial credit.
%---------------- End of the document ---------------

\section{Product Rule}

Since the behavior of derivatives of sums behave so well, its tempting to believe that the derivatives of products would behave similarly well, that is, it may be reasonable to guess that $\frac{d}{dx}[f(x)g(x)]=f'(x)g'(x)$.  To try to see if this is true, we should ask, is:

$$\frac{d}{dx}[x\cdot x]=\frac{d}{dx}[x]\frac{d}{dx}[x]?$$

Well on the left hand side $\frac{d}{dx}[x\cdot x]=\frac{d}{dx}[x^2]=2x$.  On the right, $\frac{d}{dx}[x]\frac{d}{dx}[x]=1\cdot1=1$, so these are clearly not the same.  What should it be?\\

We do have a rule for products called the {\bf product rule} and it states $\frac{d}{dx}[f(x)g(x)]=f'(x)g(x)+f(x)g'(x).$  To see why:

\begin{eqnarray*}
\frac{d}{dx}[f(x)g(x)]&=&\limit{h}{0}\frac{f(x+h)g(x+h)-f(x)g(x)}{h},\ \text{at this stage, I'm going to pull a clever trick and add by 0:}\\
&=&\limit{h}{0}\frac{f(x+h)g(x+h)\textcolor{blue}{-f(x)g(x+h)+f(x)g(x+h)}-f(x)g(x)}{h}\\
&=&\limit{h}{0}\frac{f(x+h)g(x+h)-f(x)g(x+h)}{h}+\limit{h}{0}\frac{f(x)g(x+h)-f(x)g(x)}{h}\\
&=&\limit{h}{0}\frac{f(x+h)-f(x)}{h}g(x+h)+\limit{h}{0}f(x)\frac{g(x+h)-g(x)}{h}\\
&=&f'(x)g(x)+f(x)g'(x).
\end{eqnarray*}



\section{Quotient Rule}

If we have a rule for differentiating products, we should have a rule for differentiating quotients as well, after all, quotients and products are basically the same thing.  So let $Q(x)=\frac{f(x)}{g(x)}$, and ask, what is $Q'(x)$.  So:

\begin{eqnarray*}
Q(x)&=&\frac{f(x)}{g(x)}\\
Q(x)g(x)&=&f(x),\ \text{now we can differentiate both sides}\\
\frac{d}{dx}[Q(x)g(x)]&=&f'(x),\ \text{then by the product rule:}\\
Q'(x)g(x)+Q(x)g'(x)&=&f'(x)\\
Q'(x)g(x)&=&f'(x)-Q(x)g'(x),\ \text{then recall that  $Q(x)=\frac{f(x)}{g(x)}$}\\
Q'(x)g(x)&=&f'(x)-\frac{f(x)g'(x)}{g(x)}\\
Q'(x)g(x)&=&\frac{f'(x)g(x)-f(x)g'(x)}{g(x)}\\
Q'(x)&=&\frac{f'(x)g(x)-f(x)g'(x)}{g(x)^2}\\
\end{eqnarray*}


So the {\bf quotient rule} is $\frac{d}{dx}[\frac{f(x)}{g(x)}]=\frac{f'(x)g(x)-f(x)g'(x)}{g(x)^2}$.


\section{Examples}

So, consider the derivative of $f(x)=x^3(x+1)$, which can be computed:

\begin{eqnarray*}
f'(x)&=&\frac{d}{dx}[x^3](x+1)+x^3\frac{d}{dx}[x+1]\\
&=&3x^2(x+1)+x^3(1)\\
&=&3x^3+3x^2+x^3\\
&=&4x^3+3x^2.
\end{eqnarray*}

Also note we could have computed this some other way:

\begin{eqnarray*}
f(x)&=&x^3(x+1)=x^4+x^3\\
f'(x)&=&4x^3+3x^2.
\end{eqnarray*}

Similarly, the derivative of $g(x)=\frac{x}{x^2+1}$:

\begin{eqnarray*}
g'(x)&=&\frac{(x^2+1)\frac{d}{dx}[x]-x\frac{d}{dx}[x^2+1]}{(x^2+1)^2}\\
&=&\frac{(x^2+1)(1)-x(2x)}{(x^2+1)^2}\\
&=&\frac{1-x^2}{(x^2+1)^2}\\
\end{eqnarray*}
























\end{document}
