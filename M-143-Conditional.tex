\documentclass[10pt]{article}

% The following command leaves more space between lines.  That's great
% when correcting drafts.  When you comment it out, however, the
% output looks much nicer.
%
\linespread{1.0}

\usepackage{amsmath}
\usepackage{amssymb}
\usepackage{graphicx}
\usepackage{epsfig}
\usepackage{latexsym}
\usepackage{amsthm}

\usepackage{mathrsfs}

\usepackage{multicol}



\usepackage[colorlinks,citecolor=blue]{hyperref}

\usepackage[latin1]{inputenc}

\usepackage{tikz-cd}
\usepackage{pgfplots}

%\usepackage{3dplot}

\usetikzlibrary{matrix,arrows,decorations.pathmorphing}
\usepackage{circuitikz}

\usepackage[scale=0.8]{geometry}


%\usepackage{umoline}\setlength{\UnderlineDepth}{1pt}
%\usepackage[linktocpage=true]{hyperref}

\input xy
\xyoption{all}


%\addtolength{\hoffset}{-.5in}
%\addtolength{\textwidth}{1in}
%\setlength{\parindent}{.5in}
%\setlength{\textheight}{9.5in} \setlength{\topmargin}{-2cm}


\pagestyle{myheadings}\parindent 0em




\usepackage[latin1]{inputenc}


%------------------copy and posted code from the internets-------------

%\numberwithin{equation}{section} % comment out when neccessary

\newtheorem{theorem}[equation]{Theorem}
\newtheorem{lemma}[equation]{Lemma}
\newtheorem{proposition}[equation]{Proposition}
\newtheorem{corollary}[equation]{Corollary}


\theoremstyle{definition}
\newtheorem{definition}[equation]{Definition}
\newtheorem{example}[equation]{Example}
\newtheorem{remark}[equation]{Remark}
\newtheorem{problem}[equation]{Problem}



\newcommand{\R}[1]{\mathbb{R}^{#1}}
\newcommand{\C}[1]{\mathbb{C}^{#1}}
\newcommand{\Z}[1]{\mathbb{Z}^{#1}}
\newcommand{\K}[1]{\mathbb{K}^{#1}}
\newcommand{\embed}[0]{\hookrightarrow}
\newcommand{\TT}[4]{\begin{tabular}{| c | c |}\hline $#1$ & $#2$ \\ \hline $#3$ & $#4$ \\ \hline\end{tabular}} %goddamn it
\newcommand{\partd}[2]{\frac{\partial #1}{\partial #2}}
\newcommand{\limit}[2]{\displaystyle{ \lim_{#1 \to #2}}}
\newcommand{\vectornorm}[1]{\left|\left|#1\right|\right|}
\newcommand{\Ker}[0]{\text{\textnormal{Ker}}}
\newcommand{\Hom}[0]{\text{\textnormal{Hom}}}
\newcommand{\circled}[1]{\tikz[baseline=(char.base)]{
            \node[shape=circle,draw,inner sep=2pt] (char) {#1};}}


\newcommand{\T}{\rotatebox[origin=c]{180}{$\scriptscriptstyle \perp $}}
\newcommand{\x}{\textbf{x}}
\newcommand{\y}{\textbf{y}}
\newcommand{\supp}{\text{\textnormal{supp}}}
\newcommand{\csupp}{\text{\textnormal{cosupp}}}
\newcommand{\found}{\text{\textnormal{found}}}
\newcommand{\roof}{\text{\textnormal{roof}}}

\newcommand{\bcup}{\displaystyle\bigcup}
\newcommand{\bcap}{\displaystyle\bigcap}
\newcommand{\dsum}{\displaystyle\sum}
\newcommand{\dint}{\displaystyle\int}





\begin{document}
%

{\bf Name:} \hrulefill\hrulefill\hrulefill\\
{\bf M143} \qquad \qquad \\
{\bf Conditional Statements}\\ %(look familiar??)\\
%Show all work for full/partial credit.
%---------------- End of the document ---------------



\section{Conditional Statements}

These things go by a few names, conditional statements, if-then statements.  They are statements of the form ``If $p$, then $q$."  When the condition $p$ is satisfied, it implies the statement $q$.  This is written in formal logic as $p\to q.$\\


Let's suppose that you have a child and you say to them ``If you are clean your room, I'll buy you ice-cream!"  What would make you a liar and what would make you a truth teller.\\

If your kid cleans their room, and you buy them ice-cream, then hooray, you kept your promise!  If your kid cleans their room and and you don't buy them ice-cream, then you didn't keep your promise and are a liar.\\

Now this part is sometimes tricky for students, but bear with me.  If your kid doesn't clean their room, you can do whatever you want.  You only said that you would do something if they cleaned their room, you made no promises as to what you would do if they didn't.  So now whether or not you buy them ice-cream, you've technically kept your promise.  This may make you a pushover parent, depending, but it wouldn't make you a liar.\\

So, if we let $p$=``kid cleans room" and $q$=``you buy them ice-cream"  Then the above sentence may be written as $p\to q$ which has the following truth table:

$$
\begin{array}{c|c|c}
p&q&p\to q\\
\hline
T&T&T\\
T&F&F\\
F&T&T\\
F&F&T
\end{array}
$$


As we can see ``If $p$ then $q$" is always true unless $p$ happens and then $q$ does not happen as promised.

\subsection{An equivalent statement}

Notice that $p\to q$ is only false under one circumstance.  So is the statement $p\vee q$.  Is it possible then to rewrite $p\to q$ as an ``or" statement somehow?\\

I claim that $p \to q = \sim p \vee q$.  To see this, consider:


 
$$
\begin{array}{c|c|c|c|c}
p&q&p\to q&\sim p & \sim p \vee q\\
\hline
T&T&T&F&T\\
T&F&F&F&F\\
F&T&T&T&T\\
F&F&T&T&T
\end{array}
$$

And we see that these things have the same truth values and are equivalent statements.  But, more importantly, consider what $\sim p \vee q$ would mean in our example.  ``Either you don't clean your room or you can have ice-cream."  One might rephrase this as ``You can either not clean your room or you can have ice-cream."  Isn't this basically saying ``If you clean your room you can have ice-cream?"  So that definitely sounds like an equivalent statement to me.

\subsection{Negation}

What about the negation of $p\to q$?  Well, one can try all sorts of combinations of $p, q$, but what we do know is how to negate and/or statements, and how to rewrite $p\to q$ so:

\begin{eqnarray*}
\sim(p\to q)&=&\sim(\sim p \vee q)\\
&=&\sim\sim p\wedge \sim q\\
&=&p\wedge \sim q.
\end{eqnarray*}

How does that play out in practice, again using our parenting example, $p\wedge \sim q$ is ``You cleaned your room and I'm not buying you ice-cream."  Damn.  That's basically exactly the opposite of what you said before, which is what a negation is supposed to be.  

\section{A Longer Example}

Using these facts, we can take relatively complicated statements, break them down, and negate them with relative ease.  Consider this sentence:\\

``If it is sunny and I am in a good mood, then I will take a walk."\\

So let's let:

\begin{itemize}
\item $p$=``It's Sunny"
\item $q$=``I'm in a good mood."
\item $r$=``I take a walk".
\end{itemize}

We can summarize this whole thing as $(p\wedge q)\to r$.  It's not the easiest thing to see when this statement may be true.  So, we can break this thing down as follows:

\begin{eqnarray*}
(p\wedge q)\to r&=&\sim (p\wedge q)\vee r\\
&=&(\sim p \vee \sim q) \vee r
\end{eqnarray*}

So this is an equivalent statement, and it's not much easier to see that it is true as long as it isn't sunny, you're in a bad mood, or if you take a walk. We would read this as ``Either it's not sunny, or I'm not in a good mood, or I'm taking a walk." \\ 

 If we negate this thing:



\begin{eqnarray*}
\sim ((p\wedge q)\to r)&=&\sim ((\sim p \vee \sim q) \vee r)\\
&=&\sim (\sim p \vee \sim q) \wedge \sim r\\
&=& (\sim \sim p  \wedge \sim \sim q) \wedge \sim r\\
&=&(p\wedge q)\wedge \sim r
\end{eqnarray*}

We would read this as ``It IS sunny and I'm in a good mood and for some reason I'm not going to walk." which is the opposite of what was promised before.


\section{Circuits}

Consider $$\begin{circuitikz}

\draw (0,0) to[ospst=$p$] (2,0);

\end{circuitikz}$$

This is the diagram of the ``circuit" $p$ Imagine $T$ as been the bridge open and $F$ as the bridge closed.  Can you cross over, well, yes as long as the bridge is open, i.e. as long as $p$ is true.\\

Now how about this:


Consider $$\begin{circuitikz}

\draw (0,0) to[ospst=$p$] (2,0);

\draw (2,0) to[ospst=$q$] (4,0);

\end{circuitikz}$$

Can you get to the other side?  Well, as long as both bridges are open, i.e.  $p\wedge q$.\\


How about this? $$\begin{circuitikz}

\draw (0,0) to (2,0);
\draw (2,0) to (2,1);
\draw (2,0) to (2,-1);

\draw (2,1) to[ospst=$p$] (4,1);

\draw (2,-1) to[ospst=$q$] (4,-1);

\draw(4,1) to (4,0);
\draw(4,-1) to (4,0);

\draw(4,0) to (6,0);


\end{circuitikz}$$


Can you get from one side to the other?  Yes, as long as either bridge is open we can, so this is $p\vee q$.\\

Using these circuit diagrams, we can model any logical statement.  For example $p\wedge ((q\wedge r) \vee (\sim q \wedge \sim r))$ would be:


$$\begin{circuitikz}

\draw (0,0) to[ospst=$p$] (2,0);
\draw (2,0) to (2,1);
\draw (2,0) to (2,-1);

\draw (2,1) to[ospst=$q$] (3,1);
\draw (3,1) to[ospst=$r$] (4,1);

\draw (2,-1) to[ospst=$\sim q$] (3,-1);
\draw (3,-1) to[ospst=$\sim r$] (4,-1);

\draw(4,1) to (4,0);
\draw(4,-1) to (4,0);

\draw(4,0) to (5,0);


\end{circuitikz}$$

And


$$\begin{circuitikz}

\draw (0,0) to[ospst=$p$] (2,0);
\draw (2,0) to (2,1);
\draw (2,0) to (2,-1);

\draw (2,1) to[ospst=$q$] (3,1);
\draw (3,1) to[ospst=$r$] (4,1);

\draw (2,-1) to (3,-1);
\draw (3,-1) to (3,-.5);
\draw (3,-1) to (3,-1.5);

\draw (3,-.5) to[ospst=$\sim s$] (4,-.5);
\draw (3,-1.5) to[ospst=$\sim t$] (4,-1.5);





\draw(4,1) to (4,0);
\draw(4,-1.5) to (4,0);

\draw(4,0) to (5,0);


\end{circuitikz}$$


would be $p \wedge ((q\wedge r)\vee (\sim s\vee \sim t))$.\\

Here, it again helps to be able to break everything down into ands/ors.  We don't have something for implications, but of we wanted to do the circuit for $\sim((p\vee q)\to r)$, we could note:

\begin{eqnarray*}
\sim((p\vee q)\to r)&=&\sim(\sim(p\vee q)\vee r)\\
&=&(p\vee q)\wedge \sim r.
\end{eqnarray*}

So it's circuit would be:

$$\begin{circuitikz}

\draw (0,0) to (2,0);
\draw (2,0) to (2,1);
\draw (2,0) to (2,-1);

\draw (2,1) to[ospst=$p$] (4,1);

\draw (2,-1) to[ospst=$q$] (4,-1);

\draw(4,1) to (4,0);
\draw(4,-1) to (4,0);

\draw(4,0) to[ospst=$\sim r$] (6,0);


\end{circuitikz}$$





\end{document}
