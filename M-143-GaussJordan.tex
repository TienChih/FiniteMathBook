\documentclass[10pt]{article}

% The following command leaves more space between lines.  That's great
% when correcting drafts.  When you comment it out, however, the
% output looks much nicer.
%
\linespread{1.0}

\usepackage{amsmath}
\usepackage{amssymb}
\usepackage{graphicx}
\usepackage{epsfig}
\usepackage{latexsym}
\usepackage{amsthm}

\usepackage{mathrsfs}

\usepackage{multicol}



\usepackage[colorlinks,citecolor=blue]{hyperref}

\usepackage[latin1]{inputenc}

\usepackage{tikz-cd}
\usepackage{pgfplots}

%\usepackage{3dplot}

\usetikzlibrary{matrix,arrows,decorations.pathmorphing}


\usepackage[scale=0.8]{geometry}


%\usepackage{umoline}\setlength{\UnderlineDepth}{1pt}
%\usepackage[linktocpage=true]{hyperref}

\input xy
\xyoption{all}


%\addtolength{\hoffset}{-.5in}
%\addtolength{\textwidth}{1in}
%\setlength{\parindent}{.5in}
%\setlength{\textheight}{9.5in} \setlength{\topmargin}{-2cm}


\pagestyle{myheadings}\parindent 0em




\usepackage[latin1]{inputenc}


%------------------copy and posted code from the internets-------------

%\numberwithin{equation}{section} % comment out when neccessary

\newtheorem{theorem}[equation]{Theorem}
\newtheorem{lemma}[equation]{Lemma}
\newtheorem{proposition}[equation]{Proposition}
\newtheorem{corollary}[equation]{Corollary}


\theoremstyle{definition}
\newtheorem{definition}[equation]{Definition}
\newtheorem{example}[equation]{Example}
\newtheorem{remark}[equation]{Remark}
\newtheorem{problem}[equation]{Problem}



\newcommand{\R}[1]{\mathbb{R}^{#1}}
\newcommand{\C}[1]{\mathbb{C}^{#1}}
\newcommand{\Z}[1]{\mathbb{Z}^{#1}}
\newcommand{\K}[1]{\mathbb{K}^{#1}}
\newcommand{\embed}[0]{\hookrightarrow}
\newcommand{\TT}[4]{\begin{tabular}{| c | c |}\hline $#1$ & $#2$ \\ \hline $#3$ & $#4$ \\ \hline\end{tabular}} %goddamn it
\newcommand{\partd}[2]{\frac{\partial #1}{\partial #2}}
\newcommand{\limit}[2]{\displaystyle{ \lim_{#1 \to #2}}}
\newcommand{\vectornorm}[1]{\left|\left|#1\right|\right|}
\newcommand{\Ker}[0]{\text{\textnormal{Ker}}}
\newcommand{\Hom}[0]{\text{\textnormal{Hom}}}
\newcommand{\circled}[1]{\tikz[baseline=(char.base)]{
            \node[shape=circle,draw,inner sep=2pt] (char) {#1};}}


\newcommand{\T}{\rotatebox[origin=c]{180}{$\scriptscriptstyle \perp $}}
\newcommand{\x}{\textbf{x}}
\newcommand{\y}{\textbf{y}}
\newcommand{\supp}{\text{\textnormal{supp}}}
\newcommand{\csupp}{\text{\textnormal{cosupp}}}
\newcommand{\found}{\text{\textnormal{found}}}
\newcommand{\roof}{\text{\textnormal{roof}}}

\newcommand{\bcup}{\displaystyle\bigcup}
\newcommand{\bcap}{\displaystyle\bigcap}
\newcommand{\dsum}{\displaystyle\sum}
\newcommand{\dint}{\displaystyle\int}





\begin{document}
%

{\bf Name:} \hrulefill\hrulefill\hrulefill\\
{\bf M143} \qquad \qquad \\
{\bf Gauss Jordan Method }\\ %(look familiar??)\\
%Show all work for full/partial credit.
%---------------- End of the document ---------------



\section{Representing a System of Linear Equations as a Matrix}

Given a system of equations, we can try to encode that information in a matrix:

So for example the system of equations:

\begin{eqnarray*}
2x+3y+z&=&3\\
-2x+4y+2z&=&0\\
x-y+2z&=&3
\end{eqnarray*}

Would have matrix representation:

$$ \left( \begin{array}{rrr|r}
2 & 3 & 1& 3\\
-2 & 4 & 2 & 0\\
1 & -1 & 2 & 3
\end{array}\right)$$

Conversely, if you have a matrix:

$$ \left( \begin{array}{rr|r}
1 & 2 & 4\\
3 & 3  & 9\\
\end{array}\right)$$

This represent the system of equations:


\begin{eqnarray*}
x+2y&=&4\\
3x+3y&=&9
\end{eqnarray*}

\section{Solving Systems of Linear Equations}

So let's consider the above system/matrix:

$$ \left( \begin{array}{rr|r}
1 & 2 & 4\\
3 & 3  & 9\\
\end{array}\right)$$

So we are allowed to manipulate this matrix in a number of ways.  Remembering that each row represents a linear equation:

\begin{enumerate}
\item {\bf You can multiply any row by a non-zero multiple}.\\ 

We can do this because say, the solutions to $x+2y=4$ are the same as the solutions to $-x-2y=-4$ or $2x+4y=8$, so we're not changing anything.

\item {\bf You can add to any row, the multiple of any row.}\\

The reason for this is basically the same as the one above.

\item {\bf You can rearrange the rows.}\\

Certainly we should believe that the solution(s) to

\begin{eqnarray*}
x+2y&=&4\\
3x+3y&=&9
\end{eqnarray*}


and

\begin{eqnarray*}
3x+3y&=&9\\
x+2y&=&4
\end{eqnarray*}

are the same.

\end{enumerate}


So our goal here is to change this system into one where it's much easier to see what the solution is.  The best way to do this, is to not have all these x's and y's, I want 2 equations: $x=?, y=??$.

So:


$$ \left( \begin{array}{rr|r}
1 & 2 & 4\\
3 & 3  & 9\\
\end{array}\right)$$
$$(1/3)R_2\mapsto R_2 \left( \begin{array}{rr|r}
1 & 2 & 4\\
1 & 1  & 3\\
\end{array}\right)$$
$$(-1)R_1+R_2\mapsto R_2 \left( \begin{array}{rr|r}
1 & 2 & 4\\
0 & -1  & -1\\
\end{array}\right)$$
$$(-1)R_2\mapsto R_2 \left( \begin{array}{rr|r}
1 & 2 & 4\\
0 & 1  & 1\\
\end{array}\right)$$
$$(-2)R_2+R_1\mapsto R_1 \left( \begin{array}{rr|r}
1 & 0 & 2\\
0 & 1  & 1\\
\end{array}\right)$$

What system of equations of this?  It is:

\begin{eqnarray*}
x+0y&=&2\\
0x+y&=&1
\end{eqnarray*}

Which must mean: $x=2, y=1$, alright!  By the way, one can test your answers to see if they are correct!

$$1(2)+2(1)=4, 3(1)+3(2)=9$$ so it checks out.


\section{Reduced Row Echelon Form}

When a matrix reaches the form above where one can just read out the solutions, this is called {\bf reduced row echelon form}.  It gives you a quick way to find all the solutions.  So lets look at the first example:


$$ \left( \begin{array}{rrr|r}
2 & 3 & 1& 3\\
-2 & 4 & 2 & 0\\
1 & -1 & 2 & 3
\end{array}\right)$$

Your first thought here is probably ``I don't want to do this."  Yeah, me neither.  One thing to note is that this process of finding a reduced row echelon form of a matrix is fairly mechanical.  So, we should be able to do this with a machine.

So, there is a link to this on my webpage under Finite Mathematics, but here's a link to a separate independent sagecell:

\url{https://sagecell.sagemath.org/?z=eJxztM1NLCnKrNAIDNRRiI420jHWMdQxjtWJ1jXSMdEx0jEAMg11dA2BTOPYWE1eroKizLwSDUcES6-oKDVNQ1MTAN6nE6M=&lang=sage}

You see the following code:

\begin{verbatim}
A=matrix(QQ, [[2,3,1,3],[-2,4,2,0],[1,-1,2,3]])
print(A)
print(A.rref())
\end{verbatim}

The first line defines the matrix $A$ with the rows identical to the rows we defined in our above matrix.  Then the code returns a printout of that matrix, and it's reduced row echelon form:


$$ \left( \begin{array}{rrr|r}
1 & 0 & 0& 1\\
0 & 1 & 0 & 0\\
0 & 0 & 1 & 1
\end{array}\right)$$

So $x=1, y=0, z=1$.  Is this right?  Well, 

\begin{eqnarray*}
2(1)+3(0)+1&=&3\\
-2(1)+4(0)+2(1)&=&0\\
1-0+2(1)&=&3
\end{eqnarray*}

Works for me.

By the way, if you enter the matrix from the earlier example, and replace the first line with:

\begin{verbatim}
A=matrix(QQ, [[1,2,4],[3,3,9]])
\end{verbatim}

You should have the reduced row echelon matrix we achieved earlier.






\section{Infinitely many solutions}

Consider the system:


\begin{eqnarray*}
x+2y+z&=&3\\
-x+y+0z&=&0\\
0x+3y+z&=&3
\end{eqnarray*}

Then by using rref using as $A$:


\begin{verbatim}
A=matrix(QQ, [[1,2,1,3],[-1,1,0,0],[0,3,1,3]])
\end{verbatim}

We would obtain:


$$ \left( \begin{array}{rrr|r}
1 & 0 & 1/3& 1\\
0 & 1 & 1/3 & 1\\
0 & 0 & 0 & 0
\end{array}\right)$$

Which gives us the system:

\begin{eqnarray*}
x+0y+(1/3)z&=&1\\
0x+y+(1/3)z&=&1\\
0x+0y+0z&=&0\\
\end{eqnarray*}

Well, the last line is useless, it basically says that $0=0$, which is true but useless.  Let's break down the remaining equations:

\begin{eqnarray*}
x+(1/3)z&=&1\\
y+(1/3)z&=&1
\end{eqnarray*}

Which can be thought of as:

\begin{eqnarray*}
x&=&1-(1/3)z\\
y&=&1-(1/3)z
\end{eqnarray*}

So basically what this is saying, no matter what $z$ is, you can fin $x$ and $y$.  So if $z=0$, then $x,y=1$. But if $z=6$, then $x,y=-1$.  Well, what is $z$?  Apparently it can be whatever we want, both of the above triplets provide perfectly legitimate solutions to the original system.  What you see described is a line of solutions.  The 3 equations describe 3 planes in 3 dimensional ($x,y,z)$ space and they intersect at this line which can be describes as the solution to this equation.

\section{No Solutions}


Consider a similar system:



\begin{eqnarray*}
x+2y+z&=&3\\
-x+y+0z&=&0\\
0x+3y+z&=&0
\end{eqnarray*}

Then by using rref using as $A$:


\begin{verbatim}
A=matrix(QQ, [[1,2,1,3],[-1,1,0,0],[0,3,1,0]])
\end{verbatim}

We would obtain:


$$ \left( \begin{array}{rrr|r}
1 & 0 & 1/3& 1\\
0 & 1 & 1/3 & 1\\
0 & 0 & 0 & 0
\end{array}\right)$$

Which gives us the system:

\begin{eqnarray*}
x+0y+(1/3)z&=&1\\
0x+y+(1/3)z&=&1\\
0x+0y+0z&=&1\\
\end{eqnarray*}


Well, what does this last line mean?  It says that $0=1$.  When will this happen?  Well basically never, no choice of $x,y,z$ will ever make this true, our 3 planes do not intersect at a point or line or at all, so there is no solution.  















\end{document}
