\documentclass[10pt]{article}

% The following command leaves more space between lines.  That's great
% when correcting drafts.  When you comment it out, however, the
% output looks much nicer.
%
\linespread{1.0}

\usepackage{amsmath}
\usepackage{amssymb}
\usepackage{graphicx}
\usepackage{epsfig}
\usepackage{latexsym}
\usepackage{amsthm}

\usepackage{mathrsfs}

\usepackage{multicol}



\usepackage[colorlinks,citecolor=blue]{hyperref}

\usepackage[latin1]{inputenc}

\usepackage{tikz-cd}
\usepackage{pgfplots}

%\usepackage{3dplot}

\usetikzlibrary{matrix,arrows,decorations.pathmorphing}


\usepackage[scale=0.8]{geometry}


%\usepackage{umoline}\setlength{\UnderlineDepth}{1pt}
%\usepackage[linktocpage=true]{hyperref}

\input xy
\xyoption{all}


%\addtolength{\hoffset}{-.5in}
%\addtolength{\textwidth}{1in}
%\setlength{\parindent}{.5in}
%\setlength{\textheight}{9.5in} \setlength{\topmargin}{-2cm}


\pagestyle{myheadings}\parindent 0em




\usepackage[latin1]{inputenc}


%------------------copy and posted code from the internets-------------

%\numberwithin{equation}{section} % comment out when neccessary

\newtheorem{theorem}[equation]{Theorem}
\newtheorem{lemma}[equation]{Lemma}
\newtheorem{proposition}[equation]{Proposition}
\newtheorem{corollary}[equation]{Corollary}


\theoremstyle{definition}
\newtheorem{definition}[equation]{Definition}
\newtheorem{example}[equation]{Example}
\newtheorem{remark}[equation]{Remark}
\newtheorem{problem}[equation]{Problem}



\newcommand{\R}[1]{\mathbb{R}^{#1}}
\newcommand{\C}[1]{\mathbb{C}^{#1}}
\newcommand{\Z}[1]{\mathbb{Z}^{#1}}
\newcommand{\K}[1]{\mathbb{K}^{#1}}
\newcommand{\embed}[0]{\hookrightarrow}
\newcommand{\TT}[4]{\begin{tabular}{| c | c |}\hline $#1$ & $#2$ \\ \hline $#3$ & $#4$ \\ \hline\end{tabular}} %goddamn it
\newcommand{\partd}[2]{\frac{\partial #1}{\partial #2}}
\newcommand{\limit}[2]{\displaystyle{ \lim_{#1 \to #2}}}
\newcommand{\vectornorm}[1]{\left|\left|#1\right|\right|}
\newcommand{\Ker}[0]{\text{\textnormal{Ker}}}
\newcommand{\Hom}[0]{\text{\textnormal{Hom}}}
\newcommand{\circled}[1]{\tikz[baseline=(char.base)]{
            \node[shape=circle,draw,inner sep=2pt] (char) {#1};}}


\newcommand{\T}{\rotatebox[origin=c]{180}{$\scriptscriptstyle \perp $}}
\newcommand{\x}{\textbf{x}}
\newcommand{\y}{\textbf{y}}
\newcommand{\supp}{\text{\textnormal{supp}}}
\newcommand{\csupp}{\text{\textnormal{cosupp}}}
\newcommand{\found}{\text{\textnormal{found}}}
\newcommand{\roof}{\text{\textnormal{roof}}}

\newcommand{\bcup}{\displaystyle\bigcup}
\newcommand{\bcap}{\displaystyle\bigcap}
\newcommand{\dsum}{\displaystyle\sum}
\newcommand{\dint}{\displaystyle\int}





\begin{document}
%

{\bf Name:} \hrulefill\hrulefill\hrulefill\\
{\bf M143} \qquad \qquad \\
{\bf Limits and Continuity}\\ %(look familiar??)\\
%Show all work for full/partial credit.
%---------------- End of the document ---------------

\section{Average rates of change}

Consider a rocket that is launched into the air and has height $f(x)=-5x^2+20x$ meters in $x$ seconds. \url{https://www.desmos.com/calculator/1rwsdsbcm9}. How fast is the rocket traveling when it's launched ($x=0$ seconds)?\\

Turns out, it's not so easy to think about what this is, we don't typically think about speeds in terms of instantaneous action.  Our methods of measuring speed reflects this, it's always stuff like meters {\bf PER SECOND}, miles {\bf PER HOUR}, in other words speed is some change in distance {\bf OVER TIME} and that period of time is not typically 0.\\

So what now?  Lets start by asking an easier question, what is the average speed of the rocket over the first 3 seconds?\\

It's best in a scenario like this to NOT overthink the underlying problem.  At 0 seconds, you're $f(0)=-5(0)^2+20(0)=0$ meters off the ground.  After 3 seconds you are $f(3)=-5(3^2)+20(3)=15$ meters off the ground.  So you rose by 15 meters in 3 seconds?  You're aveerage speed must be 5 meters per second. \url{https://www.desmos.com/calculator/6k61owwnrd}.\\

In fact:  {\bf The average rate of change of a function $f(x)$ over the interval $[a,b]$ is $m=\frac{f(b)-f(a)}{b-a}$.}  This is just the slope of the line connecting $(a,f(a))$ and $(b,f(b))$.\\

Following this idea, the average velocity of our rocket over the 1st 2 seconds is: $m=\frac{f(2)-f(0)}{2-0}=\frac{20}{2}=10$ meters per second. \url{https://www.desmos.com/calculator/ihauynuflo}\\

The average velocity of our rocket over the 1st second is: $m=\frac{f(1)-f(0)}{1-0}=\frac{15}{1}=15$ meters per second. \url{https://www.desmos.com/calculator/jce0fd1bsr}\\

The average velocity of our rocket over the 1st half second is: $m=\frac{f(.5)-f(0)}{.5-0}=\frac{8.75}{.5}=17.5$ meters per second. \url{https://www.desmos.com/calculator/xpxlltf8f5}\\

The average velocity of our rocket over the 1st quarter second is: $m=\frac{f(.25)-f(0)}{.25-0}=\frac{4.6875}{.25}=18.75$ meters per second. \url{https://www.desmos.com/calculator/74kpvz17c5}\\

The average velocity of our rocket over the 1st tenth of a second is: $m=\frac{f(.1)-f(0)}{.1-0}=\frac{1.95}{.1}=19.5$ meters per second. \url{https://www.desmos.com/calculator/qyvtzmyg8k}\\

The average velocity of our rocket over the 1st hundreth of a second is: $m=\frac{f(.01)-f(0)}{.01-0}=\frac{.1995}{.1}=19.95$ meters per second. \url{https://www.desmos.com/calculator/hrawtifwdw}\\


And it does seem like these numbers are getting closer and closer to 20 meters per second as $b$ gets closer to 0.  But if I plug in $0$ for $b$, I get an undefined $0/0$ fraction.  Oh, if only there were some concept in mathematics to describe the values of a function as your inputs slowly approach a given value.........

\section{The Derivative as a Limit}

Let's be a bit more general for just a moment, suppose that we had a function $f(x)$ and we wanted to know what the instantaneous rate of change of a function was at some given $x$.  Well, if we stepped forward by $h$ units, we could measure the average rate of change of $f(x)$ over $[x, x+h]$, this would be $$\frac{f(x+h)-f(x)}{x+h-x}=\frac{f(x+h)-f(x)}{h}.$$

What we just observed is that if we let $h$ converge towards 0, that it seems as if our quotient also converges to some value.  This value is the derivative of $f(x)$ at $x$ and is denoted and defined as $$f'(x)=\limit{h}{0}\frac{f(x+h)-f(x)}{h}.$$

So how does this play out given our rocket?  Well:

\begin{eqnarray*}
f'(x)&=&\limit{h}{0}\frac{f(x+h)-f(x)}{h}\\
&=&\limit{h}{0} \frac{-5(x+h)^2+20(x+h)-(-5x^2+20x)}{h}\\
&=&\limit{h}{0}\frac{-5x^2-10xh-5h^2+20x+20h+5x^2-20x}{h}\\
&=&\limit{h}{0}\frac{-10xh-5h^2+20h}{h}\\
&=&\limit{h}{0}\frac{(-10x-5h+20)h}{h}\\
&=&\limit{h}{0} -10x-5h+20\\
&=&-10x+20.
\end{eqnarray*}


So this means that in $0$ seconds, the change in the height was $f'(0)=-10(0)+20=20$ meters/second as we suspected!  But now we can compute all the other instantaneous rates of change as well.  So at $t=1$ second, it'll be $f'(1)=-10(1)+20=10$ meters per second.  In 2 seconds we are at the top with $f'(2)=0$ meters per second, at 3 seconds, $f'(3)=-10$ meters/second to reflect the negative change in height at this point.  Finally in 4 seconds, the rocket hits the ground at $f'(4)=-20$ meters per second.  (drag $a$ around and see for yourself! \url{https://www.desmos.com/calculator/3po0qezgel})






























\end{document}
